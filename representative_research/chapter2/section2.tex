\section{在线负面口碑处理专家特征}\label{sec:2.2}
\subsection{领域知识水平}\label{subsection:2.2.1}
社会化媒体平台上的专家用户传统上指代在某一领域具有一定程度的专业知识,
并对该领域的抱怨问题具有较多数量或较高质量的回答。
这种描述可被归纳为用户的领域知识水平(Domain Knowledge Level),
即用户提供某一领域知识层面解决方案的能力。
专家用户应当首先具备一定的领域知识水平才能满足抱怨问题提出者和在线负面口碑传播者对解决方案的需求。
为衡量领域内用户知识水平,本文将用户在该领域内的单条回答记录视作单个文本,
将用户的回答记录集合汇总为用户知识文档(user knowledge document)。
通过聚集平台上所有非重复的被采纳或标记有用回答,
构建该平台的领域知识文档(domain knowledge document),用来代表平台内的领域整体知识。

本文通过向量空间模型计算用户知识文档与领域知识文档的相似度,得出每个用户的领域知识水平。
该模型将文档视作若干词的组合,将单个词视作一个空间维度,该词的权重制视作其有向值,
由此将文档结构化为由若干词和各词权重值组成的词空间,两个文档的相似度视为词空间的相似度。
具体地,设文档集$D=\{d_1,d_2,\ldots,d_N \}$,文档$d_i \in D, i=(1,2,\ldots,N)$,
可用$d_i=\{(t_{i,1},w_{i,1} ),(t_{i,2},w_{i,2} ),\ldots,(t_{i,N},w_{i,N} )\}$表示,
其中$t_{i,j}$为文档$d_i$中的第$j$个词$w_{i,j}$,为词$t_{i,j}$在文档$d_i$中的权重值,$j=(1,2,\ldots,N)$。
本文采用TF-IDF方法计算词的权重值$w_{i,j}$,
表示为:
\begin{align}\label{eq:2.1}
    w_{i,j} &=tf_{i,j} \times idf_j \\ 
    &= \frac{num_{i,j}}{\sum^N_{j=1} num_{i,j}} \times \ln \frac{|D|}{|\{ i\colon t_{i,j} \in d_i \}|} \notag
\end{align}

其中$tf_{i,j}$为词$t_{i,j}$在文档$d_i$中出现的频率,
表示为该词在文档$d_i$中出现的次数$num_{i,j}$与文档$d_i$中所有词出现次数之和的商;
$idf_j$为词$t_{i,j}$的逆向文件频率,
表示为文档集$D$内的文件总数$|D|$与包含该词之文件数$|\{ i\colon t_{i,j} \in d_i \}|$商的对数。

根据\autoref{eq:2.1}可知文档内词$t_{i,j}$的权重值$w_{i,j}$,
由此可计算用户知识文档与领域知识文档的相似度,即本文所定义用户的领域知识水平$DKL$,表示为:
\begin{equation}\label{eq:2.2}
    DKL_i= Sim_{(ukd_i,dkd)}=\frac{\sum^N_{j=1} w_{ { ukd_{i,j} } } w_{dkd_j} }{\sqrt{\sum^N_{j=1} w^2_{ { ukd_{i,j} } } } \sqrt{\sum^N_{j=1} w^2_{ { dkd_j } } }}
\end{equation}

其中$Sim_{(ukd_i,dkd)}$表示用户知识文档$ukd_i$与领域知识文档$dkd$的相似度,
$w_{ { ukd_{i,j} } }$为用户知识文档$ukd_i$中第$j$个词的权重值,
$w_{ { dkd_j } }$为领域知识文档中第$j$个词的权重值,$j=(1,2,\ldots,N)$。

\subsection{情感状态}\label{subsection:2.2.2}

在线负面口碑中包含了发布和传播者的负面情绪,专家用户在满足其知识需求的同时,
应当具有抚慰其情感的能力,即正面的情感状态(Sentiment State)。
回答中积极正面的情感能有效抑制抱怨行为中的负面情绪,一方面是因其回答带有一定的安抚,
如礼貌、幽默等能有效缓解在线负面口碑发布和传播者的情绪;
另一方面带有如真诚、同情等积极正面情感的回答更易被用户接受。
因此这种正面情感状态对负面情绪的抑制会有效提高处理在线负面口碑的成功率\cite{kim2007best}。

社区平台中用户表达情绪的主要方式是文本,
本文通过使用情感词典计算用户文档的平均情感得分,以此表示用户的情感状态。
情感得分的计算以情感极性(Sentiment Polarity)和情感强度(Sentiment Intensity)为基础,
本文使用知网的中文情感词典HowNet作为参考分析文档中的情感极性和情感强度,
具体包括正面和负面情感词语词典、正面和负面评价词语词典和程度级别词语词典。
其中,正面和负面词典是情感极性分析的基础;程度级别词语虽没有情感倾向,但会影响文本的情感强度。
其中否定词词典未被HowNet收纳且本身没有情感倾向,但否定词会改变情感极性的方向,
故本文在HowNet的基础上扩充了否定词词典。扩充后的情感词典如\autoref{tab:2.1}所示。
\begin{table}[ht]
    \centering
    \caption{情感词典}\label{tab:2.1}
    \vskip -10pt
    \begin{tabularx}{\textwidth}{ccYY}
    \toprule
    词性描述 & 词汇实例 & 权值 & 个数\\
    \midrule
    正面情感词语 & 爱、乐于、巴不得、叹为观止、…… & 1 & 836\\
    负面情感词语 & 哀、不爽、不待见、不是滋味、…… & -1 & 1254 \\
    正面评价词语 & 棒、便利、不可缺、别具匠心、…… & 1 & 3730 \\
    负面评价词语 & 暗、昂贵、不成熟、爱理不理、…… & -1 & 3116 \\
    程度级别词语(最) & 最、十足、倍加、绝对、充分、…… & 9 & 69 \\
    程度级别词语(很) & 好、不少、分外、出奇、格外、…… & 7 & 42 \\
    程度级别词语(较) & 多、更加、还要、较为、那么、…… & 5 & 37 \\
    程度级别词语(稍) & 怪、稍稍、略微、有些、有点、…… & 3 & 29 \\
    程度级别词语(欠) & 微、相对、不大、半点、轻度、…… & 1 & 12 \\
    否定词词语 & 不、没、别、不够、没有、…… & -1 & 19 \\
    \bottomrule
    \end{tabularx}
\end{table}

此情感词典将正面和负面词语分为正面情感词语、负面情感词语、正面评价词语和负面评价词语,
并将其作为情感极性词语,其极性分别记为1、-1、1和-1;
将词语的程度级别分为5级,并将其作为情感强度词语,其权值用1、3、5、7、9表示,从1至9情感强度逐渐增强;
否定词的权值用-1表示,并将其和程度级别词语一同作为情感强度词语对情感极性词语进行修饰。

本文在计算用户文档的情感极性$SP$和情感强度$SI$时,对文档$d_i$进行分句分词处理,将所得词与情感词典中的词进行匹配,
获取文档语句内的情感极性词语和情感强度词语。通过计算文档内词语情感极性与情感强度乘积之和得出文档的平均情感得分,即本文所定义的情感状态$SS$,表示为:
\begin{equation}\label{eq:2.3}
    SS_i= \frac{\sum^N_{j=1} SP_{d_i , j} SI_{d_i , j} }{N}
\end{equation}
其中$SP_{d_i , j} $表示文档$d_i$中第$j$个词的情感极性,
$SI_{d_i , j} $表示文档$d_i$中第$j$个词的情感强度,$N$表示文档$d_i$中的情感词数,$j=(1,2,\ldots,N)$。

\subsection{互动程度}\label{subsection:2.2.3}

在面向在线负面口碑处理的专家识别中,用户的领域知识水平和情感状态是衡量其是否具有专家能力的直接因素。
然而,在社会化媒体平台多方参与的前提下,
用户的专家能力能否有效体现与其领域知识水平和情感状态的传递广度和深度即互动程度(Social Degree)显著相关。
广泛且深入的互动能有效提升专家用户的影响范围,对在线负面口碑的传播具有抑制作用。
因此专家用户应当具有一定的互动程度以充分体现自身的专家能力。
关注与被关注数常用来说明用户的互动程度,这种特征选取在说明社会化媒体用户社交程度时具有较好的效果\cite{cha2010measuring}。
高关注与被关注数对用户互动程度有很大的提升,但在专家识别背景中,单纯通过关注与被关注关系很难证明用户对其专家能力的认同。
在处理在线负面口碑的过程中,用户首先会根据自己的情景发布抱怨,
专家用户针对此问题提出解决方案即回答,回答被用户采纳则表明用户对回答者专家能力的认可,
此时专家用户通过知识和情感的传递即互动满足了在线负面口碑发布者的需求;
具有和提问者相同经历或看法的其他用户在对抱怨进行搜索和传播时也可以对回答进行标记,
若标记为有用,则表示专家用户通过互动满足了在线负面口碑传播者的需求。
作为社会化媒体平台,平台常对被采纳或标记有用数较高的回答赋予标签(如“推荐回答”、“精华”等),
并根据被采纳或标记有用回答数计算用户累计推荐等级。
同时,平台为调动用户互动的积极性,常根据用户的在线时长、发帖和回答数等计算用户等级。
因此本文选取用户回答数、累计推荐等级和等级作为用户互动程度的指标综合反映用户领域知识水平和情感状态的传递深度和广度。
基于以上分析,本文选择领域知识水平$DKL$、情感状态$SS$和互动程度$SD$三种特征表示在线负面口碑处理的专家能力,
如\autoref{tab:2.2}所示。
\begin{table}[ht]
    \centering
    \caption{在线负面口碑处理的专家能力特征}\label{tab:2.2}
    \vskip -10pt
    \begin{tabularx}{\textwidth}{ccY}
    \toprule
    特征 & 指标 & 描述 \\
    \midrule
    领域知识水平 & 知识文档相似度 & 用户知识文档与领域知识文档的相似度\\
    情感状态 & 平均情感得分 & 用户所有文档的情感得分之和与情感词数的商 \\
    互动程度 & 回答数 & 用户的回答总数 \\
     & 累计推荐等级 & 平台根据用户被标记推荐回答数计算的等级 \\
     & 等级 &平台根据用户积极性计算的等级 \\
    \bottomrule
    \end{tabularx}
\end{table}

根据上文的专家能力描述,可将在线负面口碑处理的专家识别转化为一个二分类问题,
并训练一个有监督的高斯核支持向量机(Support Vector Machine)分类器进行分类。
分类器通过学习专家能力特征,以判定用户是否具有专家能力,从而完成专家用户的识别。

