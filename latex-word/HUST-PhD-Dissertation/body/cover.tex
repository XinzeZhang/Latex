
%%% Local Variables:
%%% mode: Xelatex
%%% TeX-master: t
%%% End:

\ctitle{标题:宋体,英文 Times New Roman,一号,加粗,不超 30 字\\
\textcolor{red}{中英文标题、学科专业、导师姓名正确、一致}}

\xuehao{D2019xxxxx} \schoolcode{10487}
\csubjectname{XXXXX} \cauthorname{XXX}
\csupervisorname{XXX} \csupervisortitle{教授}
\defencedate{202X~年~X~月~X~日} \grantdate{}
\chair{}%
\firstreviewer{} \secondreviewer{} \thirdreviewer{}

\etitle{English Title,Times New Roman,小二号,实词的首字母大写\\
\textcolor{red}{中英文标题、学科专业、导师姓名正确、一致}}
\edegree{Doctor of Engineering}
\esubject{Control Science and Engineering}
\eauthor{(中文习惯,姓在前且姓全部大写)}
\esupervisor{Prof. xxx}
\edate{May, 2022}

%定义中英文摘要和关键字
\cabstract{

摘要是学位论文极为重要、不可缺少的组成部分,它是论文的窗口,并频繁用于国内外资料交流、情报检索、二次文献编辑等。其性质和要求如下:\par
    [1] 摘要即摘录论文要点,是论文要点不加注释和评论的一篇完整的陈述性短文,具有很强的自含性和独立性,能独立使用和被引用。\par
    [2] 博士学位论文的摘要应包含全文的主要信息,并突出创造性成果。\par
    [3] 内容范围应包含以下基本要素:\par
    \hspace{1em}(1) 目的:研究、研制、调查等的前提、目的和任务以及所涉及的主题范围。\par
    \hspace{1em}(2) 方法:所用原理、理论、条件、对象、材料、工艺、手段、装备、程序等。\par
    \hspace{1em}(3) 结果:实验的、研究的、调查的、观察的结果、数据,被确定的关系,得到的效果、性能等。\par
    \hspace{1em}(4) 结论:结果的分析、研究、比较、评价、应用;提出的问题,今后的课题,建议,预测等。\par
    \hspace{1em}(5) 其他:不属于研究、研制、调查的主要目的,但就其见识和情报价值而言也是重要的信息。\par
    [4] 摘要的详简度视论文的内容、性质而定,\textcolor{red}{博士学位论文摘要一般为800-1000汉字。}\par
    [5] 摘要及全文中均不得出现“我们”等字样。一般不用图、表、化学结构式、计算机程序,不用非公知公用的符号、术语和非法定的计量单位。\par
    [6] 摘要中一般不使用缩写词,若实在需要,在第一次使用前,需给出中文全称(缩写词);在使用英文缩写词之前,需给出中文全称(英文全称,缩写词),再次出现时可以采用中文或英文缩写词。\par
    [7] \textcolor{red}{关键词应有3至8个,另起一行置于摘要下方,领域从大到小排列。关键词之间用分号隔开,最后一个关键词后面无标点。}\par
    [8] 摘要、关键词采用中文宋体;英文Times New Room;小四号;}


\ckeywords{关键词1;关键词2;关键词3}

\eabstract{This is abstract. 

英文摘要字体为Times New Roman,小四,1.5倍行距。

\textcolor{red}{英文摘要和关键词应与中文相对应。}英语摘要用词应准确,使用本学科通用的词汇;摘要中主语(作用)常常省略,因而一般使用被动语态;应使用正确的时态,并要注意主、谓语的一致,必要的冠词不能省略。}

\ekeywords{Keyword1, Keyword2, Keyword3}
