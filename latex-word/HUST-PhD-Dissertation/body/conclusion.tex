%%% mode: latex
%%% TeX-master: t
%%% End:

\chapter{总结与展望}
\label{cha:conclusion}

\section{本文主要内容及结论}
\label{sec:conclusion}
对全文进行全面地总结,并根据各章节归纳出若干有机联系的论点。按正文的内容分段描述,包括本研究“做了什么(提出**新理论/算法、设计或研发**工艺/仪器)、获取什么结果、得出什么结论”。

请特别注意,全文总结与摘要及各章的小节要有所区分,不能简单的拷贝。这里的重点是结论,结论应该准确、完整、明确、精练。

\textcolor{red}{(论文总结与摘要及各章的小节有区分,不简单拷贝。总结包含学位论文的创新点。)}


\section{本文主要创新点}
\label{sec:contribution}
通常情况下,学位论文的创新点应放在最后一章。

创新点要凝炼,表述要清晰明了,如提出了什么创新的思路,主要特点是什么,相比现有理论或技术的提高是什么、或者有什么新的发现,是否具有重要的科学意义或应用前景。既不能过于简单,也不要太细。

硕士学位论文创新点不宜太多,一般为2个左右即可,要注意归纳创新点,千万不要以为越多越好。论文的创新不以创新点的多少来评定的,而以其创新性的价值来评定。几章的工作合在一起凝炼成一个创新点也不是不可以的。


\section{展望}
\label{sec:futurework}
对本研究成果的意义、推广应用的现实性或可能性加以论述。同时,描述本文研究中尚存在的不足,或因时间尚未完成但又必须继续的工作,对进一步的工作进行展望。